% !TeX program = xelatex
% !TeX encoding = UTF-8

\documentclass[a4paper]{article}

% 导入必要的包
\usepackage{geometry}
\usepackage{fontspec}
\usepackage{xeCJK}
\usepackage{indentfirst}
\usepackage{titlesec}
\usepackage{setspace}
\usepackage{microtype}
\usepackage{zhnumber} % 用于中文数字
\usepackage{hyperref} % 超链接

% 设置页面边距
\geometry{
  top=2.54cm,
  bottom=2.54cm,
  left=3.18cm,
  right=3.18cm
}

% 设置中文字体
\usepackage[UTF8]{ctex}

% 设置英文字体为Times New Roman
\setmainfont{Times New Roman}

% 设置标题格式
\titleformat{\section}
  {\centering\fontsize{24pt}{36pt}\selectfont\bfseries}
  {}
  {0em}
  {}

% 定义中文编号计数器
\newcounter{ChineseSection}
\renewcommand{\theChineseSection}{\chinese{ChineseSection}}

\titleformat{\subsection}
  {\fontsize{16pt}{24pt}\selectfont\bfseries}
  {\stepcounter{ChineseSection}\theChineseSection、}
  {0.5em}
  {}

% 调整标题间距
\titlespacing*{\subsection}
  {0pt}  % 左边距
  {0.5ex plus 0.2ex minus 0.1ex}  % 上方间距
  {0.5ex plus 0.2ex minus 0.1ex}  % 下方间距

% 设置正文字体大小和行距
\renewcommand{\normalsize}{\fontsize{14pt}{21pt}\selectfont}

% 设置段落首行缩进
\setlength{\parindent}{2em}

% 文档开始
\begin{document}

% 正文内容
\section{标题}

% 称呼
\noindent 尊敬的Prof.:

您好!

我是【姓名】,今年【年龄】岁,目前就读于【本科院校】【学院】【专业】。现在将本人的基本情况介绍如下。

\subsection{政治表现}

【此处介绍政治思想表现、党组织关系、参与的党团活动等内容】

【此处介绍担任的学生工作、社会工作经历等】

【此处介绍社会实践、志愿服务等经历】

\subsection{外语水平}

【此处介绍外语考试成绩、外语应用能力等】

\subsection{业务和科研能力}

【此处介绍专业课程学习情况、GPA、专业排名等】

【此处介绍获得的学术奖项、奖学金等】

【此处介绍参与的科研项目、成果、发表的论文等】

【此处介绍掌握的专业技能、软件工具等】

\subsection{研究计划}

【此处介绍研究生阶段的学习和研究计划】

【第一年计划】

【第二年计划】

【第三年计划】

\subsection{总结}

【此处是个人总结,表达对目标院校的向往和自己的决心】

% 落款
\begin{flushright}
  【姓名】\\
  \today
\end{flushright}

\end{document}